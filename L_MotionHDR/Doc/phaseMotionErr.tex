\documentclass{article}

\usepackage{times,cite}
\usepackage{epsfig}
\usepackage{graphicx}
\usepackage{amsmath}
\usepackage{amssymb}
\usepackage{caption,subcaption}
\usepackage{float}
\usepackage{multirow,comment}
\usepackage{pifont}
\usepackage{soul,hyperref}
\usepackage{enumerate}
\usepackage{enumitem}
\usepackage{color}
\newcommand{\nc}{\textcolor{red}}

% define \umu for upright mu (eg. to abbreviate micrometer)
\DeclareFontFamily{U}{euc}{}% I chose euc because the chart is called Euler cursive
\DeclareFontShape{U}{euc}{m}{n}{<-6>eurm5<6-8>eurm7<8->eurm10}{}%
\DeclareSymbolFont{AMSc}{U}{euc}{m}{n} % I chose AMSc because AMSa and AMSb are defined in the amsfonts-package
\DeclareMathSymbol{\umu}{\mathord}{AMSc}{"16}
\DeclareMathOperator*{\argmin}{arg\,min}

\newcommand\independent{\protect\mathpalette{\protect\independenT}{\perp}}
\def\independenT#1#2{\mathrel{\rlap{$#1#2$}\mkern2mu{#1#2}}}

\begin{document}
% 2014.10.24
%%%%%%%%% TITLE
\title{Phase Motion under Multiple Constraints}
\author{
Donglai \and Neal \\
}

\maketitle
\bibliographystyle{../../../Util/splncs}
\bibliography{../ref}
\section{Notation}
\begin{tabular}{|l|l|}
\hline
2D Motion&$u=\left(\begin{array}{c}u^1\\u^2\end{array}\right)$\\
\hline 
Two frames& $I_1(x)$, $I_2=I_1(x+u)$\\
\hline 
Amplitude and phase& $(A,\phi)$\\
\hline 
Scale and rotation& $(s,\theta)$\\
\hline 
\end{tabular}
\section{Phase Measurement}
We here derive the Gaussian distribution of the noisy phase measurement at the scale $s_i$ and orientation $\theta_i$.
\subsection{Noiseless Phase Measurement $\phi_0$}
Given image $I$, we apply the steered $G_2^{\theta_i},H_2^{\theta_i}$ filter and get response
\begin{align}
f_g = G_2^{\theta_i} \ast I,\ \ \ \  
f_h = H_2^{\theta_i} \ast I
\end{align}
The ground truth amplitude $A_0$ and the phase $\phi_0$ can be calculated as
\begin{align}
A_0 = \sqrt{f_g^2+f_h^2},\ \ \ \  
\phi_0 = \arctan{\frac{f_h}{f_g}}
\end{align}

\subsection{Noisy Phase Measurement $\mathcal{P}(\phi)$}
We assume the image is corrupted by Gaussian white noise $n=\mathcal{N}(0,\sigma_n^2)$ at each pixel $(x,y)$ 
\begin{align}
I_n(x,y) = I(x,y)+n\sim\mathcal{N}(I(x,y),\sigma_n^2),
\end{align}
and we get the distribution of the filter response $f_g,f_h$:
\begin{align}
\mathcal{P}(f_g,f_h)=  
&=\mathcal{P}(G_2^{\theta_i}\ast I_n,H_2^{\theta_i}\ast I_n)\nonumber\\ 
&=\mathcal{P}(G_2^{\theta_i}\ast n,H_2^{\theta_i}\ast n)\nonumber\\ 
&\sim\mathcal{N}(\vec 0, \left( \begin{array}{cc} G_2^{\theta_i}G_2^{\theta_iT}\sigma_n^2&H_2^{\theta_i}G_2^{\theta_iT}\\G_2^{\theta_i}H_2^{\theta_iT}&H_2^{\theta_i}H_2^{\theta_iT}\sigma_n^2\end{array}\right))
\end{align}
From Hilbert transform, we know $G_2^{\theta_i}H_2^{\theta_iT}=0, G_2^{\theta_i}G_2^{\theta_iT}=H_2^{\theta_i}H_2^{\theta_iT}$. We let $\sigma_0^2=G_2^{\theta_i}G_2^{\theta_iT}$ and simplify Eq.\nc{(4)} as 
\begin{align}
\mathcal{P}(f_g,f_h) = \mathcal{N}(\vec 0,\sigma_0^2\sigma_n^2\cdot I)
\end{align}
Changing to polar coordinate, we get the distribution of amplitude and phase measurement $A,\phi$:
\begin{align}
\mathcal{P}(A,\phi)&= =\frac{1}{2\pi\sigma_n^2,\sigma_0^2}\exp\left[-\frac{A_0^2\sin^2(\phi-\phi_0)}{2\sigma_n^2\sigma_0^2}\right]\exp\left[-\frac{(A-A_0\cos(\phi-\phi_0))^2}{2\sigma_n^2\sigma_0^2}\right]A
\end{align}
We set $\beta=\frac{A_0}{\sigma_n^2\sigma_0^2}$ and integrate out the amplitude $A$:
\begin{align}
\mathcal{P}(\phi)&=\frac{1}{2\pi}\exp\left[-\frac{\beta^2}{2}\right]
+\frac{\beta\cos(\phi-\phi_0)}{\sqrt{8\pi}}
(1+erf\left(\frac{\beta\cos(\phi-\phi_0)}{\sqrt{2}}\right))
\exp\left[-\frac{\beta^2\sin^2(\phi-\phi_0)}{2}\right]
\end{align}
Since we care about small motions, where $\phi-\phi_0\rightarrow 0$, 
we use the following Taylor expansion at $\phi_0$:
\begin{align}
\sin(\phi-\phi_0)&=(\phi-\phi_0)+o(\phi-\phi_0), \nonumber\\ 
\cos(\phi-\phi_0)&=1+o(\phi-\phi_0)\nonumber,
\end{align}
and the approximated distribution for noisy phase measurement is a Gaussian distribution
\begin{align}
\mathcal{P}(\phi)&\approx\frac{1}{2\pi}\exp\left[-\frac{\beta^2}{2}\right]
+\frac{\beta}{\sqrt{8\pi}}
(1+erf\left(\frac{\beta}{\sqrt{2}}\right))
\exp\left[-\frac{\beta^2(\phi-\phi_0)^2}{2}\right]\nonumber\\
&\sim \exp\left[-\frac{(\phi-\phi_0)^2}{2\frac{\sigma_n^4\sigma_0^4}{A_0^2}}\right]
\end{align}
\section{From Phase Measurement to Velocity Measurement}
\subsection{One Phase Measurement}
We have two methods:
\begin{enumerate}
\item Constant Wavelength: the projected velocity $u_{\theta_i}$ is related with the phase change by the wavelength $\lambda$ 
\begin{align}
u_{\theta_i}=\frac{\lambda}{2\pi}\phi_t\nonumber,
\end{align}
and it follows the Gaussian distribution,
which constrains the original velocity $u$
\begin{align}
(\cos\theta_i,\sin_i)\left(\begin{array}{c}u^1\\u^2\end{array}\right) =u_{\theta_i}\sim\mathcal{N}(\mu_i,\sigma_i^2)
\end{align}
\item Instantaneous Wavelength:
We adaptively estimate the center wavelength with the phase texture.
Or we can apply the phase constancy assumption and obtain the equation for the motion in the projected direction $u_{\theta_i}$ (Lucas-Kanada):
\begin{align}
(\phi_x, \phi_y)\left(\begin{array}{c}u^1\\u^2\end{array}\right) = -\phi_t \nonumber
\end{align}
Since $\phi\sim\mathcal{N}$, $u$ follows the ratio of Gaussian distributions, which is the Cauchy distribution. 
To simplify the model, we plug in $(\hat\phi_x,\hat\phi_y)$, the MAP estimation of $(\phi_x,\phi_y)$ 
and 
\begin{align}
(\hat\phi_x, \hat\phi_y)\left(\begin{array}{c}u^1\\u^2\end{array}\right)
=-\phi_t\sim\mathcal{N}(\mu_i,\sigma_i^2)
\end{align}
\end{enumerate}
For both methods, we obtain a 1D Gaussian distribution for a linear combination of $(u^1,u^2)$
\begin{align}
c_i = (a_i,b_i)\left(\begin{array}{c}u^1\\u^2\end{array}\right)  \sim \mathcal{N}(\mu_i,\sigma_i^2)
\end{align}

\subsection{Multiple Phase Measurements}
Assume the observations are independent,
we can rewrite the log joint likelihood:
\begin{align}
\log L&\propto \sum_i \frac{(a_iu^1+b_iu^2-\mu_i)^2}{2\sigma_i^2}\nonumber\\
&\propto \sum_i(\frac{a_i}{\sigma_i}u^1+\frac{b_i}{\sigma_i}u^2-\mu_i/\sigma_i)^2\nonumber\\
&\propto
\left( \begin{array}{c} u^1\\u^2 \end{array} \right)^T
\left( \begin{array}{cc} \sum_i\frac{a_i^2}{\sigma_i^2}&\sum_i\frac{a_ib_i}{\sigma_i^2}\\\sum_i\frac{a_ib_i}{\sigma_i^2}&\sum_i\frac{b_i^2}{\sigma_i^2}\end{array} \right)
\left( \begin{array}{c} u^1\\u^2 \end{array} \right)
-2\left( \begin{array}{c} \sum_i\frac{a_i\mu_i}{\sigma_i^2}\\\sum_i\frac{b_i\mu_i}{\sigma_i^2} \end{array} \right)^T
\left( \begin{array}{c} u^1\\u^2 \end{array} \right)\nonumber\\
&\propto 
\left( (\begin{array}{c} u^1\\u^2 \end{array})-B_2^{-1}C_2 \right)^T
B_2
\left( (\begin{array}{c} u^1\\u^2 \end{array})-B_2^{-1}C_2 \right)
\end{align}
where $B_2=\left( \begin{array}{cc} \sum_i\frac{a_i^2}{\sigma_i^2}&\sum_i\frac{a_ib_i}{\sigma_i^2}\\\sum_i\frac{a_ib_i}{\sigma_i^2}&\sum_i\frac{b_i^2}{\sigma_i^2}\end{array} \right), C_2=\left( \begin{array}{c} \sum_i\frac{a_i\mu_i}{\sigma_i^2}\\\sum_i\frac{b_i\mu_i}{\sigma_i^2} \end{array} \right)$.
Thus we have
\begin{align}
\left( \begin{array}{c} u^1\\u^2 \end{array} \right)
&\sim \mathcal{N}(B_2^{-1}C_2,B_2^{-1})\\
&\sim \mathcal{N}(\frac{\left( \begin{array}{c} \sum_{i>j}\frac{(a_ib_j-a_jb_i)(b_i\mu_j-b_j\mu_i)}{\sigma_i^2\sigma_j^2}\\\sum_{i>j}\frac{(a_ib_j-a_jb_i)(a_i\mu_j-a_j\mu_i)}{\sigma_i^2\sigma_j^2} \end{array} \right)}{\sum_{i>j}\frac{(a_ib_j-a_jb_i)^2}{\sigma_i^2\sigma_j^2}},
\frac{\left( \begin{array}{cc} \sum_i\frac{b_i^2}{\sigma_i^2}&-\sum_i\frac{a_ib_i}{\sigma_i^2}\\-\sum_i\frac{a_ib_i}{\sigma_i^2}&\sum_i\frac{a_i^2}{\sigma_i^2}\end{array} \right)}{\sum_{i>j}\frac{(a_ib_j-a_jb_i)^2}{\sigma_i^2\sigma_j^2}},
)
\end{align}

\noindent\textbf{Connection to Bill's write-up}\\
For each observation, we only obtain a 1D Gaussian distribution
for a linear combination of the 2D velocity, 
which can be written as a 2D Gaussian distribution whose 
variance is $\infty$ along one direction (singular precision matrix).
\begin{align}
\frac{(a_iu^1+b_iu^2-\mu_i)^2}{2\sigma_i^2}
&=
\frac{(\left( \begin{array}{c} u^1\\u^2 \end{array} \right)^T
\left( \begin{array}{c} a_i\\b_i \end{array} \right)-\mu_i)^2}{2\sigma_i^2}\nonumber\\
&=
\frac{
\left( \begin{array}{c} u^1\\u^2 \end{array} \right)^T
\left[
\left( \begin{array}{c} a_i\\b_i\end{array} \right)^T
\left( \begin{array}{c} a_i\\b_i\end{array} \right)
\right]
\left( \begin{array}{c} u^1\\u^2 \end{array} \right)
+C_1u^1+C_2u^2+C_3
}{2\sigma_i^2}
\end{align}
To get an intuitive visualization as in Bill's write-up,
where each observation puts a 2D Gaussian blob for the velocity,
we can either truncate the variance in such direction to a large enough number $N$,
or adding a small diagonal matrix to the precision matrix.
\begin{align}
\left( \begin{array}{c} a_i\\b_i\end{array} \right)^T
\left( \begin{array}{c} a_i\\b_i\end{array} \right)
&\rightarrow
\left( \begin{array}{c} a_i\\b_i\end{array} \right)^T
\left( \begin{array}{c} a_i\\b_i\end{array} \right)
+\nc{\frac{1}{N}
\left( \begin{array}{c} -b_i\\a_i\end{array} \right)^T
\left( \begin{array}{c} -b_i\\a_i\end{array} \right)
}\nonumber\\
&\rightarrow
\left( \begin{array}{cc} a_i\\b_i\end{array} \right)^T
\left( \begin{array}{c} a_i\\b_i\end{array} \right)
+\nc{\frac{1}{N}
\left( \begin{array}{cc} 1&0\\0&1\end{array} \right)
}\nonumber
\end{align}
When $N\rightarrow\infty$, it converges to the derivation above.

\end{document}




